\documentclass[varwidth]{standalone}
\usepackage[utf8]{inputenc}
\usepackage[T1]{fontenc}
\usepackage{graphicx}
\usepackage{amsmath}
\usepackage[american,siunitx]{circuitikz}
\usetikzlibrary{tikzmark,arrows,shapes,calc,positioning}

\NewDocumentCommand\MyArrow{O{0pt}mmmO{out=150,in=210}}
{%
\begin{tikzpicture}[overlay, remember picture]
  \draw [-,thick,line width=2pt, red]
    ( $ ({pic cs:#2}) $ ) to 
    ( $ (pic cs:#3) + (0,0) $ );
\end{tikzpicture}%
}
    
\begin{document}


  \MyArrow{ll}{ur}{red}
\MyArrow{lr}{ul}{red}

\begin{tikzpicture}[overlay, remember picture]
  \node[red, align=left] at ( $ ({pic: ur}) + (5.56,-2) $) {Parámetros,\\ no variables};
\end{tikzpicture}%

\begin{itemize}
\item \tikzmark{ul} displacamiento (masa total) - 7.6 t \tikzmark{ur}
\item masa de cada ala - 1.2 t
\item altura del mástil - 28m
\item área de vela - 235 sqm
\item \tikzmark{ll} profundidad máxima con alas - 5m \tikzmark{lr}
\end{itemize}


\end{document}
